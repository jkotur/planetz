\subsection{Silnik graficzny}\label{sub:silnik graficzny}
\paragraph{}

Głównym wymaganiem względem silnika graficznego była wydajność. Miał on w wydajny sposób wyświetlić scenę składającą się z kilkunastu tysięcy planet. Nie wszystkie planety musiały być widoczne w pełnej okazałości - niektóre mogły znajdować się daleko od kamery, bądź nawet za kamerą. Aby osiągnąć ten cel, wszystkie obliczenia musiały odbywać się na karcie graficznej, ponieważ koszt transferu danych pomiędzy CPU a GPU jest zbyt wysoki.

\paragraph{}

Aktualnie udało się uzyskać bardzo wydajny silnik, którego główną wadą jest to, że służy tylko do wyświetlania kul. Jednak swoje zadanie wykonuje bardzo dobrze. Wśród zalet można wymienić:

\begin{itemize}
\item Bardzo dobra skalowalność ze względu na ilość planet.
\item Dobra reakcja na planety, które są daleko od widza.
\item Możliwość wyświetlenia stosunkowo wielu gwiazd (źródeł światła).
\end{itemize}

\paragraph{}

Przy próbach z bardzo dużą ilością planet (rzędu 64 tysięcy) silnik graficzny dawał dobre wyniki. Renderowanie przebiegało płynnie w czasie rzeczywistym z 50 klatkami/sekundę. Dzięki temu pozostawał czas na obliczenia fizyki. Dla przypadku bliższego wymaganiom projektowym, czyli dla układu posiadającego 15 tysięcy planet, silnik osiąga 120 klatek/sekundę, co jest w zupełności wystarczającym wynikiem. Testy były prowadzone na układzie będącym kostką z planet o wymiarach odpowiednio 40 i 25 planet w każdym wymiarze. W każdej kostce było kilkadziesiąt gwiazd. Kamera była ustawiona tak, by objąć widokiem wszystkie planety. W ten sposób sprawdzona została zarówno reakcja silnika na planety znajdujące się blisko obserwatora, na te które znajdowały się daleko oraz na liczne gwiazdy znajdujące się w układzie. Jest to układ, który dobrze przybliża średni przypadek.

\paragraph{}

Wszystkie powyższe testy zostały wykonane na karcie graficznej NVIDIA 9800GTX+. Była ona najsilniejszą kartą swoich czasów, jednak od lat jej młodości na rynku kart graficznych wiele się zmieniło. Aktualnie procesory graficzne posiadają nawet do pięciu razy więcej procesorów strumieniowych.

\paragraph{}

Podsumowując, silnik graficzny spełnił w zupełności nasze oczekiwania pod względem wydajności. Również efekty wizualne są zadowalające.


\subsection{Silnik fizyczny}\label{sub:silnik fizyczny}
\subsubsection{Wydajność}
Jednym z założeń aplikacji było osiągnięcie dużej wydajności. Podczas testów na sprzęcie:
\begin{itemize}
\item{Procesor Intel Core i7 930 2.8GHz}
\item{Karta graficzna nVidia GeForce GTX 275, 896MB, 240 rdzeni CUDA}
\end{itemize}
obliczenia w czasie rzeczywistym (minimum 20fps) odbywały się z udziałem 16 tysięcy planet. Wydajność znacząco spadała przy dużej ilości występujących kolizji - liczba fps spadała nawet dwukrotnie.

\subsubsection{Stabilność}
Przy większości układów aplikacja mogła działać przez kilkanaście godzin bez przerwy. Niestety, niektóre układy powodowały powstawanie błędnych danych w postaci wartości NaN w pozycjach planet.
