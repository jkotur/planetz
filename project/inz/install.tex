\subsection{Wymagania}\label{sub:wymagania}
\paragraph{}

Aplikacja została napisana z użyciem takich technologii jak CUDA i OpenGL. Implikuje to dość spore wymagania co do karty graficznej wymaganej do uruchomienia programu. Ponieważ użyliśmy wyżej wymienione komponenty w wersjach, CUDA 2.3 oraz OpenGL 3.2, karta graficzna musi być co najmniej NVIDIA geforce serii 8, lub nowsza. Jednak aby móc uruchomić program z pełnymi możliwościami graficznymi, oraz dla dużych układów planet zalecana jest karta serii 200, lub któraś z mocniejszych kart serii 9.

\paragraph{}

Oprócz powyższych wymagań które mają swoje skutki w wymaganym sprzęcie są też inne, software'owe wymagania. Najważniejszym z nich jest platforma linux. Jednakże wszystkie użyte technologie i biblioteki są przenośne i prawdopodobnie da się bez większych problemów skompilować projekt pod platformę windows. Nawet system budowania jest wieloplatformowy i po poprawnym zainstalowaniu bibliotek, będzie w stanie stworzyć pliki projektu popularnych IDE takich jak Eclipse, czy Visual Studio. Ponieważ jednak w naszym laboratorium dostępna jest platforma linux, ograniczyliśmy się do niej.

\paragraph{}

Systemem tym jest cmake. Zdecydowaliśmy się na niego głównie z powodu łatwości użytkowania, jednak wielka przenośność też nie była bez znaczenia.

\paragraph{}

Kolejnymi software'owymi wymaganiami są biblioteki użyte w programie. Naszym zadaniem było jedynie stworzenie modułów fizycznego i graficznego od podstaw, natomiast do wszelkich innych funkcjonalności programu mogliśmy użyć bibliotek zewnętrznych. Poza wyżej wymienionymi technologiami, do stworzenia projektu zostały użyte poniższe biblioteki:

\begin{description}
\item[SDL] (Simple DirectMedia Layer) -- biblioteka służąca to stworzenia kontekstu OpenGL oraz za zarządzanie oknem i wszelkimi peryferiami, potrzebna nam jest wraz z rozszerzeniem SDL\_image które pozwala na czytanie licznych formatów obrazów
\item[Boost] -- wszechstronna biblioteka do C++, służy nam między innymi do tworzenia sygnałów, obsługi delegatów i przeglądania katalogów
\item[CEGUI] (Crazy Eddie's Gui) -- biblioteka tworząca graficzny interfejs użytkownika przy pomocy OpenGLa
\item[Sqlite3] -- pozwala na obsługę baz danych sqlite3 z poziomu kodu C
\item[cudpp] -- biblioteka zawierające prymitywy algorytmów CUDA, wszystkie jej źródła dołączone są do projektu, tak że nie trzeba instalować jej oddzielnie
\item[CppUnit] -- testy jednostkowe dla C++, biblioteka konieczna jedynie do kompilacji testów
\end{description}

\subsection{Kompilacja}\label{sub:kompilacja}
\paragraph{}

Aby aplikacja mogła się zbudować potrzebne są wszystkie wyżej wymienione biblioteki. W projekcie zawarte są również testy jednostkowe, które wymagają biblioteki CppUnit do poprawnego zbudowania. Nie jest to jednak konieczne do zbudowania aplikacji. Po poprawnym zainstalowaniu bibliotek, wystarczy użyć systemu budowania. Do projektu dołączone są skrypty które powinny automatycznie znaleźć wszystkie potrzebne biblioteki, czasem konieczne może być podanie tych ścieżek ręcznie. Dla platformy linux nie powinno być jednak takiej potrzeby. Na platformie linux najłatwiej jest wygenerować natywne pliki budowania Makefile, a następnie przy ich użyciu zbudować projekt. Standardowy schemat użycia przedstawiony jest poniżej. Wszystkie komendy należy wykonywać w katalogu nadrzędnym projektu.

\begin{verbatim}
$ cmake .
$ make inz
\end{verbatim}

ewentualnie aby zbudować testy należy wpisać, już po wygenerowaniu Makefile'i:

\begin{verbatim}
$ make tests
\end{verbatim}

Po zakończeniu komendy pliki binarne znajdują się w katalogu \texttt{./bin/}.

