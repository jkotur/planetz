\paragraph{}
Program umożliwia zapis oraz odczyt symulacji. Zdecydowaliśmy się na użycie do tego celu baz danych sqlite. Są to zwykłe pliki binarne, do których można się dostać przy użyciu biblioteki sqlite3\cite{sqlite3}. Główną zaletą tego podejścia był fakt, że można je w łatwy sposób edytować i czytać, co bardzo pomogło w debugowaniu. Zajmują co prawda około trzykrotnie więcej miejsca, ale jest to rozmiar akceptowalny (ok. 30 MB dla 100 tysięcy planet).

\paragraph{}
Mechanizm zapisu w kategoriach bazodanowych był dość prymitywny - zapis oznaczał utworzenie na nowo całej tabeli planet, a odczyt pobranie wszystkich jej wierszy. Bardziej skomplikowane zapytania przydawały się jednak, gdy trzeba było zmodyfikować plik po zmianach w kodzie. Użyteczne okazały się także do "obcinania" układów do interesujących nas fragmentów, w celu zwiększenia szybkości konkretnego zbioru planet.

\subsection{Tabela planets}
\paragraph{}
Jest to główna tabela przechowująca informacje o planetach. W aktualnej wersji jej zawartość to:
\begin{verbatim}
CREATE TABLE planets(
	xcoord REAL, ycoord REAL, zcoord REAL,
	radius REAL, mass REAL,
	xvel REAL, yvel REAL, zvel REAL,
	model_id INT);
\end{verbatim}

Odpowiednie pola oznaczają:
\begin{description}
\item[xcoord, ycoord, zcoord] - koordynaty planety, a właściwie jej środka,
\item[radius] - promień planety,
\item[mass] - masa planety,
\item[xvel, yvel, zvel] - składowe wektora prędkości planety oraz
\item[model\_id] - identyfikator modelu planety, określającego jej wygląd oraz typ.
\end{description}

\subsection{Tabela camera}

\paragraph{}
Ta tabela zawiera tylko jeden wiersz. Posiada on 16 pól, odpowiadających komórkom macierzy przekształcenia, związanej z kamerą:
\begin{verbatim}
CREATE TABLE camera(
x0 REAL, x1 REAL, x2 REAL, x3 REAL,
x4 REAL, x5 REAL, x6 REAL, x7 REAL,
x8 REAL, x9 REAL, x10 REAL, x11 REAL,
x12 REAL, x13 REAL, x14 REAL, x15 REAL);
\end{verbatim}
