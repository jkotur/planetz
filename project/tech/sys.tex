\paragraph{Sprzęt}
Wymagania sprzętowe dotyczą głownie możliwości graficznych komputera. Komputer musi obsługiwać CUDA 2.3 oraz OpenGL 3.2. W praktyce oznacza to potrzebe posiadania komputera z kartą graficzną firmy NVIDIA co najmniej  serii 8000 lub nowszą. Oczywiście potrzebne będą także peryferia take jak: monitor, klawiatura oraz myszka.

\paragraph{Oprogramowanie}
Do włączenia aplikacji konieczny będzie komputer z systemem operacyjnym Linux. Wymagane biblioteki: 
\begin{description}
\item{OpenGL} - biblioteka graficzna odpowiedzialna za przekształcenia sceny 3D. W wersji 3.2 do standardu wprowadzone zostały shadery geometryczne, które będą niezbędne do szybkiego wyświetlania planet.
\item{CEGUI} - biblioteka wyświetlająca interfejs graficzny użytkownika przy pomocy opengla. Dziki temu wszystkie guziki mogą zagnieżdżone w oknie symulacji.
\item{SDL} - biblioteka odpowiedzialna za stworzenie okna w środowisku graficzny, i stworzenie kontekstu opengla. Zajmuje sie ona tez obsługą zdarzeń związanych z oknem.
\item{SDL\_image} - biblioteka wczytująca obrazy do bitmap. Wspiera duza ilosc formatow.
\item{CUDA}
\item{CUDPP}
\end{description}


