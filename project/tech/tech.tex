\documentclass[a4paper,titlepage,10pt]{article}

\usepackage[T1]{fontenc}
\usepackage[utf8]{inputenc}
\usepackage{polski}

\usepackage{enumerate}
\usepackage{amssymb}
\usepackage{amsmath}
\usepackage[pdftex]{graphicx}
\usepackage{tikz}
\usepackage[colorlinks=true,linkcolor=blue]{hyperref}
\usepackage{anysize}

\usepackage{lastpage}
\usepackage{fancyhdr}

\usepackage{pdflscape}

\usepackage[a4paper, top=2.5cm, bottom=2.5cm, left=2cm, right=2cm]{geometry}
\linespread{1.3}

\title{\huge Symulacja układu planetarnego na GPU\\ przy użyciu CUDA i OpenGL\\\small Dokumentacja techniczna}
\author{Daniel Kłobuszewski\and Jakub Kotur}

\begin{document}
	\maketitle
	
	\pagestyle{fancyplain}
%        \fancyhf{}
	\cfoot{\thepage/\pageref{LastPage}}

	\hfill \\
	\begin{tabular}{|p{2.5cm}|p{0.8cm}|p{2cm}|p{2cm}|p{3cm}|p{2cm}|}
	\hline
	\multicolumn{6}{|l|}{Metryka dokumentu} \\
	\hline
	Projekt & \multicolumn{2}{l|}{Symulacja układu planetarnego na GPU } &
	Firma & \multicolumn{2}{l|}{Politechnika Warszawska} \\
	&  \multicolumn{2}{l|}{przy użyciu CUDA i OpenGL} & &  \multicolumn{2}{l|}{} \\
	\hline
	Nazwa & \multicolumn{5}{l|}{Dokumentacja biznesowa} \\
	\hline
	Temat & \multicolumn{5}{l|}{Specyfikacja biznesowa projektu} \\
	\hline
	Autor & \multicolumn{5}{l|}{Daniel Kłobuszewski, Jakub Kotur} \\
	\hline
	Plik & \multicolumn{5}{l|}{business.pdf} \\
	\hline
	Nr wersji & 01 & Status & Roboczy & Data\par sporządzenia & 2010-10-09 \\
	\hline
	Streszczenie & \multicolumn{5}{p{11cm}|}{Celem dokumentu jest zdefiniowanie
		funkcjonalnych wymagań Projektu.} \\
	\hline
	Zatwierdził & \multicolumn{3}{l|}{ } &
	Data ostatniej\par modyfikacji & 2010-10-09 \\
	\hline
\end{tabular}

\newpage

	\hfill \\
	%\newpage
	\begin{figure}[h]
	\centering

\begin{tabular}{|p{.075\textwidth}|p{.1\textwidth}|p{.2\textwidth}|p{.522\textwidth}|}
	\hline
	\multicolumn{4}{|l|}{Historia zmian dokumentu} \\
	\hline
	Wersja & Data & Kto & Opis \\
	\hline
	0.1 & 2011-01-02 & Jakub Kotur &
	Określenie podstawowej struktury dokumentu \\
	\hline
	1.0 & 2011-01-04 & Daniel Kłobuszewski &
	Poprawki ortograficzne i stylistyczne \\
	\hline
\end{tabular}

	\label{tab:metric}
\end{figure}



	\newpage

	\tableofcontents
	\newpage

	\section{Streszczenie}\label{sec:streszczenie}
	\paragraph{}
Poniższy dokument, będący pracą inżynierską, stanowi podsumowanie projektu, pisanego w ramach przedmiotu Projekt Zespołowy, na wydziale Matematyki i Nauk Informacyjnych Politechniki Warszawskiej w semestrze zimowym 2010/2011.

\paragraph{}
Opisujemy w nim założenia i wymagania projektowe, zasady działania głównych modułów, różnice w stosunku do specyfikacji oraz wnioski z testów końcowych. Do dokumentu dołączona jest także krótka instrukcja obsługi programu, która pozwoli zapoznać się z nim każdemu, kto wcześniej nie miał z naszą aplikacją styczności.

\paragraph{}
Celem projektu było stworzenie atrakcyjnej wizualnie aplikacji, która pozwala na symulację dużych układów planetarnych w czasie rzeczywistym. Początkowe założenie mówiło o równoczesnym przeliczaniu minimum 10~tysięcy planet przy zachowaniu płynności animacji. Cel ten udało się zrealizować. Na obecnie dostępnym sprzęcie można z powodzeniem symulować także większe układy.

\paragraph{}
Osiągnięcie takiej wydajności było możliwe dzięki przeniesieniu praktycznie wszystkich obliczeń fizycznych na kartę graficzną. Użyliśmy w tym celu technologii nVidia CUDA\cite{cuda}. Pozwala ona na pisanie kodu zbliżonego do języka C, który następnie jest wykonywany równolegle na karcie graficznej. Dodatkowo korzystamy także z algorytmu k-means do dzielenia przestrzeni na klastry, w obrębie których liczone są dokładne oddziaływania. Do wyświetlania napisany został wyspecjalizowany silnik graficzny w OpenGL~\cite{ogl:bible}, oraz korzystający z techniki deffered renderingu. Pozwala on nie tylko wydajnie renderować duże ilości planet, ale także oświetlanie ich światłem pobliskich gwiazd i prezentowanie atmosfer na planetach.

\paragraph{}
Aplikacja pozwala na swobodne poruszanie się kamerą po wszechświecie lub - jeżeli wolimy - podążanie za wybraną przez nas gwiazdą/planetą. W przypadku zderzenia dwóch planet, powstaje jedna większa, zgodnie z zasadą zachowania pędu. Z poziomu interfejsu użytkownika możliwe jest manipulowanie symulacją - dodawanie i usuwanie planet, zapisywanie i wczytywanie układów do pliku, zmiany prędkości kamery oraz przebiegu symulacji. Możliwa jest także manipulacja ustawieniami wyświetlania. Dodatkowo można włączyć "śledzenie" planet - pozostawiają one wówczas za sobą ślad, co pozwala obejrzeć ich trajektorie. Symulację można w dowolnym momencie zatrzymać - na przykład by obejrzeć "zamrożony" układ z innej perspektywy.

	\section{Uzyte technologie}\label{sec:uzyte technologie}
	\paragraph{}

Jednym z podstawowych założeń tego projektu jest użycie technologii CUDA. Daje to nam możliwość oprogramowania karty graficznej, ale stawia też ograniczenia. Przede wszystkim jesteśmy zmuszeni używać języka C (konkretniej, C for CUDA) do części uruchamianej na procesorze graficznym. Najłatwiej używać go wspólnie z C++ na CPU. Oczywiście istnieją bindingi w innych językach (Java, Python), ale C++ jest z nich najszybszy - nie uwzględniając czystego C, które jednak nie wspiera obiektowości. 

\paragraph{}

Kod pisany na kartę graficzną nie będzie napisany w języku obiektowym, więc wszystkie klasy, które zdefiniowaliśmy dla jednostki graficznej są jedynie abstrakcją, która pośrednio się na niego przekłada. Chcemy uzyskać aplikację wydajnie korzystającą z karty graficznej, w związku z czym ułożenie "obiektów" w pamięci będzie pozwalało na szybki i wygodny dostęp do nich. Prawdopodobnie w pewnych miejsacach spowoduje to zmniejszenie czytelności kodu na rzecz prędkości jego wykonania.

	\section{Architektura projektu}\label{sec:architektura projektu}
	\subsection{Działanie programu}\label{sub:dzialanie programu}
	\paragraph{}

Podstawowymi aktywnościami projektu są moduły wyświetlania planet, oraz obliczania pozycji planet. Jednak dla działania aplikacji bardzo ważne są również inne aktywności, które mimo że prostsze w implementacji mają duży wpływ na architekturę projektu. Poniższy diagram przedstawia kolejne aktywności aplikacji podczas działania. Dodatkowo przestawiony na diagramie jest przepływ najważniejszej informacji, czyli pozycji planet.

\begin{figure}[h]
	\centering
	\includegraphics[width=0.75\textwidth]{activity.pdf}
	\caption{Diagram aktywności}
	\label{fig:activity}
\end{figure}

\paragraph{}

\begin{description}
	\item[Inicjalizacja] \hfill \\
	jest to pierwsza aktywność jaką wykonuje program. Składa się ona z inicjalizacji wszystkich modułów, w szczególności modułu graficznego i fizycznego. Zajmuje się także początkową alokacją pamięci i załadowaniem wszystkich potrzebnych zasobów, takich jak tekstury, modele oraz inne potrzebne dane.
	\item[Wczytanie z pliku] \hfill \\
	jeśli użytkownik zdecyduje zacząć symulację z wcześniej przygotowanego pliku ta akcja jest za to odpowiedzialna. Musi ona załadować potrzebne dane z pliku, oraz zapisać je, poprzez pamięć RAM jednostki głównej, do pamięci RAM jednostki graficznej. Użytkownik może również zdecydować o wczytaniu z pliku podczas już trwającej aplikacji. W takim przypadku konieczne jest również zadbanie o odpowiednie wyczyszczenie poprzedniej symulacji, tak aby żadne artefakty nie zostały w pamięci RAM ani jednostki głównej, ani graficznej.
	\item[Zapis do pliku] \hfill \\
	na życzenie użytkownika aktualny stan symulacji może zostać zapisany do pliku. W takim przypadku zbierane z aplikacji są wszystkie potrzebne dane, takie jak pozycje planet, wielkości i ciężary planet, modele planet, pozycja kamery itp., oraz zapisywane są do pliku.
	\item[Aktywna pauza] \hfill \\
	symulacja wspiera tak zwaną "aktywną pauzę". Oznacza to że na żądanie użytkownika, symulacja fizyczna jest zatrzymana, natomiast cały czas reszta aplikacji jest w pełni funkcjonalna. Oznacza to że można sterować kamerą, wczytywać/zapisywać układy planetarne, oraz dodawać, albo usuwać planety.
	\item[Symulacja fizyczna] \hfill \\
	kluczowy moduł dla działania aplikacji. Podczas każdej takiej aktywności obliczane są kolejne pozycje planet do wyświetlenia. W jednej takiej akcji może być liczone więcej niż jedna klatka fizyczna, jeśli moc obliczeniowa komputera na to pozwala.
	\item[Symulacja graficzna] \hfill \\
	ta aktywność odpowiedzialna jest w ogólności za wyświetlanie symulacji na ekran. Głównym jej zadaniem jest wyświetlanie planet na ich pozycjach wraz z towarzyszącymi im efektami graficznymi. Pozycje planet do wyświetlenia pobiera z modułu fizycznego. Jeśli natomiast jest włączona aktywna pauza i nowe pozycje nie zostały wygenerowane, moduł korzysta ze starych pozycji. Po zakończonej jednej iteracji pętli, aplikacja odczekuje chwilę, żeby nie zajmować całego procesora, oraz żeby zapewnić takie samo działanie symulacji na słabszych komputerach (gdzie czekanie może zostać pominięte)
\end{description}



	\subsection{Struktura klas}\label{sub:struktura klas}
	\paragraph{}
Ze względu na architekturę aplikacji, klasy używane na CPU oddzielone są od klas karty graficznej. Oczywiście te drugie są klasami jedynie logicznie - ze względu na konieczność używania języka C w kodzie dla GPU. Ponieważ jednak obiekty są wygodną abstrakcją, będziemy z niej korzystać w całym programie.

\subsubsection{Klasy GPU}

\begin{figure}[h]
	\centering
	\includegraphics[angle=270,width=0.8\textwidth]{class_gpu.pdf}
	\caption{Diagram klas dla GPU}
	\label{fig:class_gpu}
\end{figure}

\paragraph{}
Struktury, z których będziemy korzystać na karcie graficznej, określone są na rysunku \ref{fig:class_gpu}. Ponieważ karta graficzna służy nam zarówno do wyświetlania danych, jak i do ich przetwarzania, wydzielone są na nim dwie przestrzenie nazw. Są to:
\begin{itemize}
	\item{Phx - do operacji fizycznych}
	\item{Gfx - do operacji graficznych}
\end{itemize}

\paragraph{}
Ponadto istnieje kilka struktur wspólnych dla obu częsci.

\paragraph{GPU::Planet} zawiera informacje, z których korzystaja zarówno wyświetlanie jak i fizyka. Jest to położenie planety oraz jej promień.
\paragraph{GPU::PointsCloud} reprezentuje chmurę cząstek - jest ona obliczana dla każdej widocznej komety przez moduł fizyczny.

\paragraph{}
Operacje fizyczne odbywać się będą z wykorzystaniem dwóch dodatkowych struktur.

\paragraph{GPU::Phx::Planet} to dodatkowe informacje o każdej planecie, które są potrzebne jedynie silnikowi fizycznemu. Należą do nich prędkość oraz masa.
\paragraph{GPU::Phx::Comet} stanowi dodatkową informację o planecie. Struktura ta istnieje tylko dla obiektów będących kometami. Zawiera kilka ostatnich pozycji oraz identyfikator planety.

\paragraph{}
Do wyświetlenia planet konieczne będą informacje o teksturach, oraz o siatkach każdej z planet.

\paragraph{GPU::Gfx::Planet} zawiera indeks modelu, czyli wyglądu planetu. Dwie planety mogą mieć ten sam model.
\paragraph{GPU::Gfx::Model} definiuje konkretny wygląd. Na tym poziomie będziemy rozróżniać zwykłe planety od gwiazd i komet.
\paragraph{GPU::Gfx::Shape} agreguje informację o siatce planety w 3D oraz odpowiadającej jej siatce na dwuwymiarowej teksturze.

\subsubsection{Klasy CPU}

\begin{figure}[ht!]
	\centering
	\includegraphics[angle=0,width=\textwidth]{class_cpu.pdf}
	\caption{Diagram klas dla CPU}
	\label{fig:class_cpu}
\end{figure}

\paragraph{}
Ta część klas przełoży się bezpośrednio na klasy znane z c++. Diagram klas znajduje się na rysunku \ref{fig:class_cpu}.

\paragraph{CPU::App} jest główną klasą, zarządzającą obiektami GFX, PHX, DataFlowMgr oraz UI. Tworzy je ona na początku działania programu.
\paragraph{CPU::GFX} odpowiada za wyświetlanie. Korzysta przy tym z biblioteki OpenGL.
\paragraph{CPU::PHX} uruchamia kernel'e CUDA, które przeprowadzają wszystkie obliczenia fizyczne.
\paragraph{CPU::DataFlowMgr} wykonuje wszystkie przepływy danych - pomiędzy RAM karty graficznej, RAM komputera oraz dyskiem twardym.
\paragraph{CPU::MemMgr} na zlecenie DataFlowMgr'a przenosi dane pomiędzy kartą graficzną a RAM.
\paragraph{CPU::IOCtl} na zlecenie DataFlowMgr'a przenosi dane pomiędzy RAM a dyskiem twardym.
\paragraph{CPU::UI} odpowiada za całość interakcji z użytkownikiem.
\paragraph{CPU::GUI}, czyli graficzny interfejs użytkownika. Obsługa okienek.
\paragraph{CPU::InputDev} obsługuje klawiaturę i mysz.
\paragraph{CPU::PlanetPicker} służy do określania, która planeta została zaznaczona.

	\section{Implementowane algorytmy}\label{sec:implementowane algorytmy}
	\subsection{Algorytmy graficzne}\label{sub:algorytmy graficzne}
	\subsubsection{Oświetlenie}\label{subsub:oswietlenie}

\paragraph{}

Oświetlenie dla sceny będzie liczone przy pomocy model Phonga. Polega to na policzeniu oświetlenia dla trzech różnych rodzai światła, oraz dodaniu ich, korzystając z prawa addytywności światła. Pierwszym z nich jest światło rozproszone, obecne na całej w takim samym natężeniu. Drugim światłem jest światło . Ten rodzaj światła uwzględnia położenie oświetlenia względem powierzchni. Pozwala to na uzyskanie realistycznych matowych powierzchni. Trzecim rodzajem jest światło odbite. Uwzględnia ono zarówno położenie wierzchołka względem światła, jak i pozycji obserwatora. Dzięki temu można uzyskać realistyczne efekty powierzchni gładkich i błyszczących.

\paragraph{}

W zależności od możliwości obliczeniowych, model oświetlenia może mieć kilka ograniczeń. Oświetlenie można obliczać dla każdego wierzchołka, natomiast kolor ściany interpolować liniowo pomiędzy kolorami wierzchołków na który składa się dana ściana. Można też obliczać natężenie światła dla każdego piksela z którego składa się ściana. Oczywistym jest że w drugim przypadku obliczenia są znacząco większe niż w przypadku pierwszym. Ograniczeniom podlega również ilość świateł obecnych na scenie. Dla każdego światła trzeba powtórzyć obliczenia dla wszystkich wierzchołków na scenie. Aby zachować płynność symulacji, ilość świateł, a zatem gwiazd, będzie ograniczona przez pewną stałą liczbę ustaloną na podstawie testów w fazie implementacji.

\subsubsection{Efekt atmosfery}\label{subsub:efekt atmosfery}

\paragraph{}

Aby uzyskać efekt atmosfery otaczającej planety, można zastosować dwa odmienne podejścia. Decyzja które z nich zostanie wybrane, pozostawiona została do fazy implementacji, ponieważ zależy to mocno od wydajności całego silnika graficznego.

\paragraph{}

Efekty atmosferyczne mogą być generowane przy pomocy techniki post-processingu. Polega to na nakładaniu obrazu, na już uzyskany obraz w trakcie normalnego renderowania sceny 3D. W przypadku atmosfer, należy wygenerować lekko rozmyte okręgi w miejscach gdzie wyrenderowane zostały planety z odpowiednim promieniem.

\paragraph{}

Drugim rozwiązaniem jest otoczenie planet półprzezroczystymi sferami, o promieniu trochę większym niż promień planet. Rozwiązanie ma tą zaletę, że całym procesem wyświetlania zajmuje się biblioteka graficzna. Minusem może być wydajność, choć nie jest to konieczne.


\subsubsection{Dynamiczna szczegółowość planet}\label{subsub:dynamiczna szczegolowosc planet}

\paragraph{}

W zależności od odległości od obserwatora, planety powinny być wyświetlane z różnym poziomem szczegółowości. Rozwiązaniem będzie wyświetlanie modelu kuli, o mniejszej ilości wierzchołków. Wszystkie modele ładowane będą na początku z pliku, do pamięci ram karty graficznej, skąd będą później wyświetlane, w zależności od tego który model będzie aktualnie potrzebny.

\subsubsection{Komety}\label{subsub:komety}

\paragraph{}

Warkocz komety jest typowym przykładem efektu cząsteczkowego. Dla chmury wygenerowanych cząsteczek, wyświetlana jest niewielka tekstura z kamieniem, bądź ogniem. Dzięki dużej ilości takich tekstur, powstanie realistyczny obraz płomienia ciągnącego się za kometą.


	\subsection{Algorytmy fizyczne}\label{sub:algorytmy fizyczne}
	\subsubsection{Obliczanie pozycji planet}

\paragraph{}

Obliczanie nowych pozycji planet odbywać się będzie iteracyjnie. Dla każdej pary planet w danym klastrze, zostanie obliczona siła oddziałująca na każdą z nich, co po podzieleniu przez masę da przyspieszenie. Mówi o tym znany z pierwszych lekcji fizyki wzór:

\begin{displaymath}
	F = ma
\end{displaymath}
\begin{displaymath}
	a = \frac{F}{m}
\end{displaymath}

Przyspieszenie jest pochodną prędkości po czasie, mamy więc

\begin{displaymath}
	a = \frac{\delta V}{\delta t}
\end{displaymath}

co w warunkach komputerowych można uprościć jako przyrost

\begin{displaymath}
	a = \frac{\Delta V}{\Delta t}
\end{displaymath}
\begin{displaymath}
	\Delta V = a\Delta t
\end{displaymath}

\paragraph{}
Analogicznie wyliczamy zmianę prędkości każdej planety przez wpływ pozostałych klastrów (sumaryczne masy oraz środki ciężkości trzymamy obliczone dla każdego klastra). Po obliczeniu nowej prędkości

\begin{displaymath}
	V_{new} = V_{old} + \Delta V
\end{displaymath}

możemy uaktualnic pozycje planet, dodając do punktu położenia wektor przesunięcia planety

\begin{displaymath}
	x_{new} = x_{old} + V \Delta t
\end{displaymath}

\subsubsection{Klasteryzacja}

\paragraph{}

Do pogrupowania planet w klastry użyty zostanie algorytm k-means. W algorytmie tym na zmianę znajdujemy środki klastrów, po czym przyporządkowujemy punkty do najbliższego środka. K-means bardzo dobrze nadaje się do naszych celów, biorąc pod uwagę, iż zwykle planety i tak będą zgrupowane wokół masywniejszych gwiazd. Ułatwi to i przyspieszy pracę algorytmu, gdyż zwykle wybór punktów początkowych ma kluczowy wpływ na czas działania oraz wynik. Jako początkowe środki klastrów będzie można wziąć środki k najmasywniejszych obiektów. Wyznaczenie wartości k, a zatem ilości klastrów, będzie wykonywane na podstawie ilości oraz mas klastrowanych obiektów.

Zasada działania algorytmu k-means została zapisana poniżej w pseudokodzie:

\begin{enumerate}
	\item{Wybierz k punktów przestrzeni, będą to środki klastrów}
	\item{Każdy z punktów do klasteryzacji przydziel do klastra, którego środek jest najbliższy}
	\item{Wyznacz błąd kwantyzacji: \ensuremath{D = {1\over{n}}\sum_{i = 1}^{n}d(x_i, r)}, gdzie \ensuremath{r} jest środkiem klastra, do którego należy \ensuremath{x_i} }
	\item{Jeżeli \ensuremath{\frac{\Delta{D}}{D}\geqslant\epsilon}, wyznacz nowe środki klastrów (jako średnie z punktów w tych klastrach) i przejdź do punktu 2.}
\end{enumerate}

\subsubsection{Kolizje}

\paragraph{}

W przypadku kolizji, dwie kolidujące ze sobą planety mogą zmienić się w jeden lub więcej nowych obiektów. Maksymalna ilość nowych obiektów jest ograniczona w ustawieniach aplikacji. Faktyczna ilość nowych obiektów, ich typy, wielkości, masy oraz prędkości wynikać będą pośrednio (typy, ilość, wielkość) bądź bezpośrednio (masy, prędkości) z praw fizyki. W szczególności przestrzegać należy prawa zachowania pędu. Suma mas także nie może się zmienić. Promienie planet oraz ich ilość będą losowane, z uwzględnieniem wielkości fizycznych. W szczególności obiekt o małej masie nie może rozbić obiektu o dużo większej masie, a z dwóch planet o dużej gęstości nie zrobi się gazowy olbrzym o wielkim promieniu.

\paragraph{}

Samo wykrywanie kolizji jest trywialne w obrębie klastra - gdzie i tak obliczamy wpływ każdy-z-każdym. Teoretycznie może to prowadzić do "przeniknięcia" się planet z różnych klastrów, ale w momencie gdy dwa obiekty będą się mijać, prawdopodobieństwo, że będą znajdować się w różnych klastrach jest niewielkie.

\subsubsection{Dodatki}\label{alg:additions}

\paragraph{}
Jeżeli harmonogram pozwoli, rozwiązywanie kolizji pomiędzy planetami z różnych klastrów będzie zaimplementowane dokładnie, poprzez wprowadzenie drzewiastego podziału przestrzeni. Każdy poziom drzewa będzie dzielił przestrzeń klastrami. W wyniku takiej klasteryzacji łatwo będzie odcinać grupy planet jako niemożliwe do skolidowania. Jest to jednak skomplikowane implementacyjnie i niezbyt widoczne dla użytkownika. Decyzja o włączeniu tej funkcjonalności zapadnie więc na etapie implementacji.

	\section{Powtórne uzycie}\label{sec:powtorne uzycie}
	\paragraph{}

Z uwagi na użyte technologie oraz specyfikę projektu, możliwości ponownego użycia kodu będą ograniczone. Zarówno silnik graficzny, jak i silnik fizyczny, są mocno wyspecjalizowane dla aktualnego projektu. Do ponownego użytku najlepiej nadaje się algorytm klasteryzacji, ponieważ rozwiązuje on standardowy problem n-ciał. Pozostałe algorytmy fizyczne są raczej nieprzydatne w innych projektach. Analogicznie jest przy silniku graficznym, gdzie tylko model oświetlenia jest problemem standardowym. Pozostałe efekty wykorzystują znane techniki, ale do wygenerowania konkretnych efektów, na potrzeby tego projektu.


	\section{Testowanie}\label{sec:testy}
	\subsection{Silnik graficzny}\label{sub:silnik graficzny}
\paragraph{}

Głównym wymaganiem względem silnika graficznego była wydajność. Miał on w wydajny sposób wyświetlić scenę składającą się z kilkunastu tysięcy planet. Nie wszystkie planety musiały być widoczne w pełnej okazałości - niektóre mogły znajdować się daleko od kamery, bądź nawet za kamerą. Aby osiągnąć ten cel, wszystkie obliczenia musiały odbywać się na karcie graficznej, ponieważ koszt transferu danych pomiędzy CPU a GPU jest zbyt wysoki.

\paragraph{}

Aktualnie udało się uzyskać bardzo wydajny silnik, którego główną wadą jest to, że służy tylko do wyświetlania kul. Jednak swoje zadanie wykonuje bardzo dobrze. Wśród zalet można wymienić:

\begin{itemize}
\item Bardzo dobra skalowalność ze względu na ilość planet.
\item Dobra reakcja na planety, które są daleko od widza.
\item Możliwość wyświetlenia stosunkowo wielu gwiazd (źródeł światła).
\end{itemize}

\paragraph{}

Przy próbach z bardzo dużą ilością planet (rzędu 64 tysięcy) silnik graficzny dawał dobre wyniki. Renderowanie przebiegało płynnie w czasie rzeczywistym z 50 klatkami/sekundę. Dzięki temu pozostawał czas na obliczenia fizyki. Dla przypadku bliższego wymaganiom projektowym, czyli dla układu posiadającego 15 tysięcy planet, silnik osiąga 120 klatek/sekundę, co jest w zupełności wystarczającym wynikiem. Testy były prowadzone na układzie będącym kostką z planet o wymiarach odpowiednio 40 i 25 planet w każdym wymiarze. W każdej kostce było kilkadziesiąt gwiazd. Kamera była ustawiona tak, by objąć widokiem wszystkie planety. W ten sposób sprawdzona została zarówno reakcja silnika na planety znajdujące się blisko obserwatora, na te które znajdowały się daleko oraz na liczne gwiazdy znajdujące się w układzie. Jest to układ, który dobrze przybliża średni przypadek.

\paragraph{}

Wszystkie powyższe testy zostały wykonane na karcie graficznej NVIDIA 9800GTX+. Była ona najsilniejszą kartą swoich czasów, jednak od lat jej młodości na rynku kart graficznych wiele się zmieniło. Aktualnie procesory graficzne posiadają nawet do pięciu razy więcej procesorów strumieniowych.

\paragraph{}

Podsumowując, silnik graficzny spełnił w zupełności nasze oczekiwania pod względem wydajności. Również efekty wizualne są zadowalające.


\subsection{Silnik fizyczny}\label{sub:silnik fizyczny}
\subsubsection{Wydajność}
Jednym z założeń aplikacji było osiągnięcie dużej wydajności. Podczas testów na sprzęcie:
\begin{itemize}
\item{Procesor Intel Core i7 930 2.8GHz}
\item{Karta graficzna nVidia GeForce GTX 275, 896MB, 240 rdzeni CUDA}
\end{itemize}
obliczenia w czasie rzeczywistym (minimum 20fps) odbywały się z udziałem 16 tysięcy planet. Wydajność znacząco spadała przy dużej ilości występujących kolizji - liczba fps spadała nawet dwukrotnie.

\subsubsection{Stabilność}
Przy większości układów aplikacja mogła działać przez kilkanaście godzin bez przerwy. Niestety, niektóre układy powodowały powstawanie błędnych danych w postaci wartości NaN w pozycjach planet.

	\section{Ograniczenia czasowe}\label{sec:ograniczenia czasowe}
	\paragraph{}

Projekt ten stanowi pracę inżynierską, przez co ma ściśle określone ramy czasowe. Ostatecznym terminem oddania projektu jest 5 stycznia 2010 roku. Ze względu na ograniczenia czasowe nie zaimplementujemy wszystkich funkcjonalności, które wymyśliliśmy. Niektóre, jak na przykład podane w \ref{alg:additions}, zostaną włączone, o ile czas na to pozwoli. Niniejsza dokumentacja zawiera opisy funkcjonalności, które muszą się znaleźć w projekcie.

	\section{Wymagania systemowe}\label{sec:wymagania systemowe}
	\paragraph{Sprzęt}
Wymagania sprzętowe dotyczą głownie możliwości graficznych komputera. Komputer musi obsługiwać CUDA 2.3 oraz OpenGL 3.2. W praktyce oznacza to potrzebe posiadania komputera z kartą graficzną firmy NVIDIA co najmniej  serii 8000 lub nowszą. Oczywiście potrzebne będą także peryferia take jak: monitor, klawiatura oraz myszka.

\paragraph{Oprogramowanie}
Do włączenia aplikacji konieczny będzie komputer z systemem operacyjnym Linux. Wymagane biblioteki: 
\begin{description}
\item{OpenGL} - biblioteka graficzna odpowiedzialna za przekształcenia sceny 3D. W wersji 3.2 do standardu wprowadzone zostały shadery geometryczne, które będą niezbędne do szybkiego wyświetlania planet.
\item{CEGUI} - biblioteka wyświetlająca interfejs graficzny użytkownika przy pomocy opengla. Dziki temu wszystkie guziki mogą zagnieżdżone w oknie symulacji.
\item{SDL} - biblioteka odpowiedzialna za stworzenie okna w środowisku graficzny, i stworzenie kontekstu opengla. Zajmuje sie ona tez obsługą zdarzeń związanych z oknem.
\item{SDL\_image} - biblioteka wczytująca obrazy do bitmap. Wspiera duza ilosc formatow.
\item{CUDA}
\item{CUDPP}
\end{description}



	\section{Uwagi}\label{sec:uwagi}
	\input{notes.tex}

\end{document}

