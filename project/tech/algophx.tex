\subsubsection{Obliczanie pozycji planet}

\paragraph{}

Obliczanie nowych pozycji planet odbywać się będzie iteracyjnie. Dla każdej pary planet w danym klastrze, zostanie obliczona siła oddziałująca na każdą z nich, co po podzieleniu przez masę da przyspieszenie. Mówi o tym znany z pierwszych lekcji fizyki wzór:

\begin{displaymath}
	F = ma
\end{displaymath}
\begin{displaymath}
	a = \frac{F}{m}
\end{displaymath}

Przyspieszenie jest pochodną prędkości po czasie, mamy więc

\begin{displaymath}
	a = \frac{\delta V}{\delta t}
\end{displaymath}

co w warunkach komputerowych można uprościć jako przyrost

\begin{displaymath}
	a = \frac{\Delta V}{\Delta t}
\end{displaymath}
\begin{displaymath}
	\Delta V = a\Delta t
\end{displaymath}

\paragraph{}
Analogicznie wyliczamy zmianę prędkości każdej planety przez wpływ pozostałych klastrów (sumaryczne masy oraz środki ciężkości trzymamy obliczone dla każdego klastra). Po obliczeniu nowej prędkości

\begin{displaymath}
	V_{new} = V_{old} + \Delta V
\end{displaymath}

możemy uaktualnic pozycje planet, dodając do punktu położenia wektor przesunięcia planety

\begin{displaymath}
	x_{new} = x_{old} + V \Delta t
\end{displaymath}

\subsubsection{Klasteryzacja}

\paragraph{}

Do pogrupowania planet w klastry użyty zostanie algorytm k-means. W algorytmie tym na zmianę znajdujemy środki klastrów, po czym przyporządkowujemy punkty do najbliższego środka. K-means bardzo dobrze nadaje się do naszych celów, biorąc pod uwagę, iż zwykle planety i tak będą zgrupowane wokół masywniejszych gwiazd. Ułatwi to i przyspieszy pracę algorytmu, gdyż zwykle wybór punktów początkowych ma kluczowy wpływ na czas działania oraz wynik. Jako początkowe środki klastrów będzie można wziąć środki k najmasywniejszych obiektów. Wyznaczenie wartości k, a zatem ilości klastrów, będzie wykonywane na podstawie ilości oraz mas klastrowanych obiektów.

Zasada działania algorytmu k-means została zapisana poniżej w pseudokodzie:

\begin{enumerate}
	\item{Wybierz k punktów przestrzeni, będą to środki klastrów}
	\item{Każdy z punktów do klasteryzacji przydziel do klastra, którego środek jest najbliższy}
	\item{Wyznacz błąd kwantyzacji: \ensuremath{D = {1\over{n}}\sum_{i = 1}^{n}d(x_i, r)}, gdzie \ensuremath{r} jest środkiem klastra, do którego należy \ensuremath{x_i} }
	\item{Jeżeli \ensuremath{\frac{\Delta{D}}{D}\geqslant\epsilon}, wyznacz nowe środki klastrów (jako średnie z punktów w tych klastrach) i przejdź do punktu 2.}
\end{enumerate}

\subsubsection{Kolizje}

\paragraph{}

W przypadku kolizji, dwie kolidujące ze sobą planety mogą zmienić się w jeden lub więcej nowych obiektów. Maksymalna ilość nowych obiektów jest ograniczona w ustawieniach aplikacji. Faktyczna ilość nowych obiektów, ich typy, wielkości, masy oraz prędkości wynikać będą pośrednio (typy, ilość, wielkość) bądź bezpośrednio (masy, prędkości) z praw fizyki. W szczególności przestrzegać należy prawa zachowania pędu. Suma mas także nie może się zmienić. Promienie planet oraz ich ilość będą losowane, z uwzględnieniem wielkości fizycznych. W szczególności obiekt o małej masie nie może rozbić obiektu o dużo większej masie, a z dwóch planet o dużej gęstości nie zrobi się gazowy olbrzym o wielkim promieniu.

\paragraph{}

Samo wykrywanie kolizji jest trywialne w obrębie klastra - gdzie i tak obliczamy wpływ każdy-z-każdym. Teoretycznie może to prowadzić do "przeniknięcia" się planet z różnych klastrów, ale w momencie gdy dwa obiekty będą się mijać, prawdopodobieństwo, że będą znajdować się w różnych klastrach jest niewielkie.

\subsubsection{Dodatki}\label{alg:additions}

\paragraph{}
Jeżeli harmonogram pozwoli, rozwiązywanie kolizji pomiędzy planetami z różnych klastrów będzie zaimplementowane dokładnie, poprzez wprowadzenie drzewiastego podziału przestrzeni. Każdy poziom drzewa będzie dzielił przestrzeń klastrami. W wyniku takiej klasteryzacji łatwo będzie odcinać grupy planet jako niemożliwe do skolidowania. Jest to jednak skomplikowane implementacyjnie i niezbyt widoczne dla użytkownika. Decyzja o włączeniu tej funkcjonalności zapadnie więc na etapie implementacji.
