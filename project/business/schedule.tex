\section{Harmonogram}\label{sec:harmonogram}

\paragraph{}

Harmonogram prac przedstawiony jest poniżej wraz z opisami poszczególnych zadań. Zadania podzielone są tematycznie.

\subsection{Grafika}

\begin{description}
	\item[Wyświetlanie planet] \hfill \\
	Celem tego zadania jest stworzenie podstawowego świata 3D do wizualizacji symulacji. W skład zadania wchodzi dynamiczne generowanie modelu kuli na zadanym poziomie szczegółowości, oraz reprezentacja go w postaci zrozumialej przez jednostkę graficzną.
	\item[Teksturowanie planet] \hfill \\
	Zadanie to polega na wygenerowaniu współrzędnych tekstury zgodnych z modelem 3D planety, oraz przekazanie tekstur oraz ich współrzędnych na jednostkę graficzną.
	\item[Dynamiczna zmiana szczegółowości planet] \hfill \\
	W zależności od odległości kamery od planety, powinna być ona wyświetlana na rożnym poziomie dokładności. W szczególności, blisko kamery powinna być to możliwie dokładna kula, natomiast w dużej odległości, powinna sie stawać punktem, bądź nawet nie powinna być wyświetlana w ogóle.
	\item[efekt atmosfery] \hfill \\
	Dla bardziej realistycznego wyglądu planet, powinna je otaczać półprzezroczysta poświata wyglądająca jak atmosfera.
	\item[efekt warkocza komety] \hfill \\
	Niektóre obiekty będą kometami. Dla realistycznego wyglądu, należy wyświetlić chmurę cząsteczek wygenerowanych przez moduł fizyczny, tak aby wyglądała na ogon komety.
	\item[oświetlenie] \hfill \\
	Oprócz oświetlenia rozproszonego, wygodnego w modelu oświetlenia, punktowym źródłem światła są gwiazdy. Inne planety powinny być oświetlane od gwiazd które są dość blisko, natomiast gwiazdy będące daleko powinny być pomijane przy obliczaniu koloru planety. Najlepszy w tym celu będzie model oświetlenie Phonga.
	\item[efekt gwiazd] \hfill \\
	Gwiazdy są specyficznymi obiektami z punktu widzenia oświetlenia, ponieważ same emitują światło. Dla takich obiektów nie należy obliczać oświetlenia, natomiast nadać im realistyczny kolor.
\end{description}

\subsection{Interfejs uzytkownika}

\begin{description}
	\item[First] \hfill \\
	The first item
	\item[Second] \hfill \\
	The second item
	\item[Third] \hfill \\
	The third etc \ldots
\end{description}

\subsection{Symulacja fizyczna}

\begin{description}
	\item[First] \hfill \\
	The first item
	\item[Second] \hfill \\
	The second item
	\item[Third] \hfill \\
	The third etc \ldots
\end{description}

\subsection{Inne}

\begin{description}
	\item[First] \hfill \\
	The first item
	\item[Second] \hfill \\
	The second item
	\item[Third] \hfill \\
	The third etc \ldots
\end{description}
