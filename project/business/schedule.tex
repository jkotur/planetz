\section{Harmonogram}\label{sec:harmonogram}

\paragraph{}

Harmonogram prac przedstawiony jest poniżej wraz z opisami poszczególnych zadań. Zadania podzielone są tematycznie.

\subsection{Grafika}

\begin{description}
	\item[Wyświetlanie planet] \hfill \\
	Celem tego zadania jest stworzenie podstawowego świata 3D do wizualizacji symulacji. W skład zadania wchodzi dynamiczne generowanie modelu kuli na zadanym poziomie szczegółowości, oraz reprezentacja go w postaci zrozumialej przez jednostkę graficzną.
	\item[Teksturowanie planet] \hfill \\
	Zadanie to polega na wygenerowaniu współrzędnych tekstury zgodnych z modelem 3D planety, oraz przekazanie tekstur oraz ich współrzędnych na jednostkę graficzną.
	\item[Dynamiczna zmiana szczegółowości planet] \hfill \\
	W zależności od odległości kamery od planety, powinna być ona wyświetlana na rożnym poziomie dokładności. W szczególności, blisko kamery powinna być to możliwie dokładna kula, natomiast w dużej odległości, powinna sie stawać punktem, bądź nawet nie powinna być wyświetlana w ogóle.
	\item[Efekt atmosfery] \hfill \\
	Dla bardziej realistycznego wyglądu planet, powinna je otaczać półprzezroczysta poświata wyglądająca jak atmosfera.
	\item[Efekt warkocza komety] \hfill \\
	Niektóre obiekty będą kometami. Dla realistycznego wyglądu, należy wyświetlić chmurę cząsteczek wygenerowanych przez moduł fizyczny, tak aby wyglądała na ogon komety.
	\item[Oświetlenie] \hfill \\
	Oprócz oświetlenia rozproszonego, wygodnego w modelu oświetlenia, punktowym źródłem światła są gwiazdy. Inne planety powinny być oświetlane od gwiazd które są dość blisko, natomiast gwiazdy będące daleko powinny być pomijane przy obliczaniu koloru planety. Najlepszy w tym celu będzie model oświetlenie Phonga.
	\item[Efekt gwiazd] \hfill \\
	Gwiazdy są specyficznymi obiektami z punktu widzenia oświetlenia, ponieważ same emitują światło. Dla takich obiektów nie należy obliczać oświetlenia, natomiast nadać im realistyczny kolor.
\end{description}

\subsection{Interfejs uzytkownika}

\begin{description}
	\item[Sterowanie symulacją] \hfill \\
	Interfejs użytkownika powinien umożliwiać wszystkie operacje uwzględnione w diagramie przypadków użycia. Ponieważ nie ma ich durzy i są niezależne, dla każdej opcji będzie przewidziany oddzielny przycisk widoczny w głównym oknie symulacji.
	\item[Interaktywna kamera] \hfill \\
	Kamera będzie wolna. Oznacza to ze użytkownik przy pomocy myszki i klawiszy, może dowolnie poruszać sie po świecie 3D, w tym przypadku po kosmosie.
\end{description}

\subsection{Symulacja fizyczna}

\begin{description}
	\item[Obliczanie nowych pozycji planet] \hfill \\
	Silnik fizyczny musi na bieżąco, na podstawie mas planet i ich położeń względem siebie, obliczać prędkości oraz nowe położenia. 
	\item[Detekcja i obsługa kolizji] \hfill \\
	Może się zdarzyć, że dwie planety znajdą się zbyt blisko siebie. Żeby nie dochodziło do "przenikania" planet przez siebie nawzajem, trzeba rozwiązać problem kolizji. Zderzające się planety będą mogły rozpaść się na więcej obiektów lub stworzyć jeden, większy.
	\item[Klasteryzacja planet] \hfill \\
	Aby zmniejszyć ilość obliczeń pozycji planet, będą one grupowane w klastry. Planety będą oddziaływać na siebie nawzajem wewnątrz klastra. Oddziaływanie z planetami spoza klastra odbywać się będzie pośrednio, poprzez wyliczenie oddziaływań uśrednionych.

\end{description}

\subsection{Inne}

\begin{description}
	\item[Serializacja i deserializacja układów] \hfill \\
	Umożliwienie użytkownikowi na wczytywanie/zapisywanie układów planetarnych. Każdy układ będzie zapisywany do oddzielnego pliku z niezbędnymi informacjami o każdej planecie - pozycja, prędkość, masa, wielkość, rodzaj. Do opisu w pliku nie będzie użyty język XML, gdyż znacznie zwiększa on ilość miejsca potrzebną do zapisu. Dane będą zapisane binarnie.
	\item[Tworzenie i usuwanie planet] \hfill \\
	Obiekt planety będzie złożony - czym innym jest planeta dla fizyki, czym innym dla grafiki. Potrzebne będą struktury danych pozwalające na optymalne wykorzystanie pamięci RAM karty graficznej - trzeba bowiem uniknąć przesyłania danych pomiędzy pamięcią karty a RAM'em komputera.
	\item[Generacja cząsteczek] \hfill \\
	Chmury cząsteczek składające się na ogon komety także będą obliczane na karcie graficznej, przez aplikację, na podstawie jej kilku ostatnich położeń.
\end{description}
