\subsection{Silnik graficzny}\label{sub:silnik graficzny}
\paragraph{}

Silnik graficzny został napisany przy użyciu popularnej ostatnio techniki deferred rendering. Popularna jest dopiero od niedawana, ponieważ wymaga stosunkowo nowych i silnych kart graficznych. W nagrodę natomiast, pozwala ona na dużą kontrolę nad procesem renderowania, oraz pozwala na tworzenie ciekawych efektów graficznych stosunkowo łatwo. Bardzo dobrze zachowuje się również przy licznych obiektach oraz dużych przestrzeniach. W przypadku układów planetarnych mamy do czynienia z takim właśnie wariantem. Wybór tej techniki pozwolił dodatkowo na optymalizację dzięki której geometria w programie została ograniczona do niezbędnego minimum. Powoduje to oczywiście dodatkowe koszty, jednak są one o wiele mniejsze niż w przypadku standardowego podejścia forward renderingu.

\subsubsection{Deferred rendering}\label{ssub:deferred rendering}
\paragraph{}

Deferred rendering polega na dwu przejściowym generowaniu finalnego obrazu. W pierwszym przejściu, przeliczana jest geometria, a wynik tych obliczeń wpisywany jest do specjalnego bufora ekranu, nazywanego g-buforem. W g-buforze zapisywane są informacje o pozycji, normalnej, kolorze danego piksela ekranu. W zależności od danego silnika bufor ten może zawierać różne dane. W drugim przejściu, na podstawie tego bufora liczone są wszystkie efekty graficzne, takie jak oświetlenie, cienie, dla każdego piksela oddzielnie. Takie podejście pozwala na bardzo wiele, oraz jest stosunkowo wydajne, posiada jednak wady. Poniżej przedstawione są wady jak i zalety deferred renderingu.

\paragraph{Zalety}

\begin{description}
\item{Skalowalność} - ponieważ wszystkie obliczenia liczone są dla piksela, silnik ten doskonale skaluje się ze względu na dodawanie geometrii do sceny
\item{Duża kontrola} - dzięki jawnemu tworzeniu buforów ekranu, można stworzyć wiele niestandardowych efektów graficznych
\end{description}

\paragraph{Wady}

\begin{description}
\item{Wielkość buforów} - bufory ekranu zajmują bardzo dużo miejsca, przy rozdzielczości full hd może być to nawet 50MB, co dla starych kart graficznych było wielkościami ogromnymi
\item{Przezroczystość} - ponieważ obliczenia są robione tylko raz dla każdego bufora, niemożliwe jest zrobienie przezroczystości w ten sposób
\item{Dużo obliczeń} - niezależnie od sceny obliczenia są proporcjonalne do wielkości ekranu i robione dla piksela. Dla starych kart graficznych jest to o wiele za ciężkie rozwiązanie
\item{Wymagane MTR} - ponieważ buforów jest wiele, wymagana od karty jest technologia MTR, czyli multiple target render, dzięki której można wygenerować za jednym zamachem wszystkie bufory. Na starych kartach graficznych potrzebne było tyle przejść ile jest buforów, co powoduje wielokrotne obliczenia geometrii
\end{description}

\subsubsection{Konstrukcja silnika}\label{ssub:konstrukcja silnika}
\paragraph{}

Silnik jest silnikiem wielo-przejściowym. Dla wyświetlenia planet wraz z oświetleniem i teksturami, potrzebne są dwa przejścia. Jedno do wygenerowania g-bufora, drugie do wygenerowania końcowego obrazu. Ponieważ w tym podejściu niemożliwe jest uzyskani przezroczystości, aby wyświetlić półprzezroczyste atmosfery konieczne jest wyświetlenie ich oddzielnie. Są one tak samo jak planety wyświetlanie dwoma przejściami przy użyciu deferred renderingu. Cały silnik posiada więc cztery przejścia, po dwa dla obrazów planet i atmosfer. Następnie wyniki tych przejść są nakładane na siebie z uwzględnieniem przezroczystości.

\subsubsection{Dane pośrednie}\label{ssub:dane pośrednie}
\paragraph{}

Silnik graficzny potrzebuje dużego bufora pośrednich danych. Buforem tym z uwagi na wygodę używania jest dwuwymiarowa tekstura zmiennoprzecinkowa. 

\subsubsection{Generowanie geometrii}\label{ssub:generowanie geometrii}
\paragraph{}

Dzięki zastosowaniu deferred renderingu, oraz tego, że renderowanymi obiektami są jedynie kule, można ominąć standardowy proces generowania geometrii. Kula która daje zadowalający efekt i jest wyświetlana przy pomocy zestawu wierzchołków, musiała by ich mieć około tysiąca. Samo przetwarzanie takiej geometrii jest kosztowne, dodatkowo doszły do tego ograniczenia karty graficznej i geometry shaderów, które trzeba by obchodzić, narażając się na dodatkowe koszty.

\paragraph{}

Wiedząc że jedyne obiekty które chcemy wyświetlić są kulami, wiemy że obiekt taki z każdej strony wygląda tak samo. Jesteśmy również w stanie łatwo policzyć składową z kuli, mając jedynie koordynaty w osiach OX i OY na tej kuli. Dzięki tym spostrzeżeniom, można stworzyć mapę głębokości (koordynat na osi OZ) kuli. Dzięki takiej mapie, w pierwszym przebiegu, wyświetlane są jedynie płaskie kwadraty, które dzięki mapie głębokości, są w stanie do bufora ekranu przekazać poprawną informację o pozycji piksela w przestrzeni. Dodatkowo ta sama mapa stanowi mapę normalnych dla oświetlenia. Dzięki takiemu podejściu, każda planeta generuje jedynie cztery wierzchołki, a reszta geometrii jest odczytywana z tekstury.

\paragraph{}

Niestety tak pięknie by to wyglądało, gdyby zastosowany był rzut ortogonalny. W symulacjach jednak konieczne do zastosowania jest rzut projekcyjny. W takim przypadku planeta widoczna na krawędzi kamery, różni się od planety widocznej na wprost. W skrajnym przypadku, gdy widoczne jest 360°, możemy widzieć planetę zupełnie od tyłu. Dlatego konieczne są dodatkowe obliczenia, uwzględniające położenie planety względem kamery.

\subsubsection{Obliczanie oświetlenia}\label{ssub:obliczanie oświetlenia}
\paragraph{}

Obliczanie oświetlenia na podstawie buforów ekranów nie stanowi problemu. W programie zastosowane jest model oświetlenia Phonga, natomiast obliczenia prowadzone są dla każdego piksela oddzielnie. Dodatkowo w celach optymalizacyjnych, nie jest liczone światło odbite (specular), ponieważ planety mają powierzchnię chropowatą, więc i tak oświetlenie to by miało znikomy wpływ na efekt końcowy, natomiast jest dość kosztowne obliczeniowo.

\paragraph{}

W celu optymalizacji, dodatkowo do każdej gwiazdy (źródła światła) przypisany jest jej zakres świecenia. Dzięki temu oświetlenie z gwiazd, które nie oświetlają planet odległych od siebie, nie jest obliczane. W przypadku gwiazd nie jest to tak znaczna optymalizacja jak w przypadku bardzo słabych świateł, jak na przykład lampki choinkowe, jednak sprawdza się przy kilku odległych od siebie galaktykach.

\subsubsection{Nakładanie tekstur}\label{ssub:nakladanie tekstur}
\paragraph{}

Skomplikowaniu ze względu na użycie deferred renderingu uległo niestety nakładanie tekstur na planety. Tekstura, tak samo jak wspomniana wcześniej mapa normalnych, jest nakładana na kulę w zależności od koordynat na osiach OX i OY. Jednak w odróżnieniu od mapy normalnych, tekstura zależy od obrotu kamery względem planety, i z każdej strony wygląda inaczej. Pojawiły się więc dwa problemy. Obracanie koordynat tekstury, tak aby widz miał wrażenie że planeta faktycznie się obraca, oraz mapowanie koordynat kwadratu reprezentującego planetę, na koordynaty tekstury planety.

\paragraph{}

Pierwszy problem został rozwiązany poprzez przekazywanie obrotu kamery do silnika graficznego, poprzez przekazanie macierzy obrotu kamery. Następnie uzyskane koordynaty kuli, są przemnażane przez tą macierz. W ten sposób otrzymywany jest bezwzględna pozycja kuli, na którą patrzy aktualnie kamera.

\paragraph{}

Drugim problemem jest mapowanie tekstury na kulę, tak aby z koordynat kuli 3D, uzyskać koordynaty na teksturze 2D. W standardowym podejściu, stosuje się siatki tekstury, natomiast każdy trójkąt jest interpolowany na płaszczyźnie. W naszym podejściu nie ma trójkątów, trzeba więc było uzyskać ciągły sposób mapowania, bez żadnych siatek. Z pomocą przyszła kartografia, która zna liczne sposoby mapowania płaszczyzny na kulę i z powrotem. Najlepsze okazały się metody mapowania o stałej powierzchni. Oznacza to że parametrem który nie ulega zniekształceniu ze względu na położenie na kuli, jest powierzchnia. Dzięki temu po mapowaniu tekstury na kulę, zniekształcenia są najmniejsze. Najtańszą obliczeniowo z tych metod okazała się metoda sinusoidalna, która wymaga policzenia jedynie jednego sinusa.

%\subsubsection{Atmosfery}\label{ssub:atmosfery}
%\paragraph{}

\subsection{Silnik fizyczny}

\subsubsection{Klastertyzacja}

\paragraph{} Do klasteryzacji używany jest alorytm k-means. Klastry definiowane są przez dwie tablice - \ensuremath{shuffle} oraz \ensuremath{count}. Do klastra o numerze k należą te planety, których indeksy znajdują się w tablicy \ensuremath{shuffle} pod indeksami z przedziału \ensuremath{< count[k-1], count[k] )}. Przyjmujemy, że \ensuremath{count[-1] = 0}.

\paragraph{} Taka reprezentacja pozwala na łatwe korzystanie z planet z danego klastra w module fizycznym i jest transparentna dla modułu graficznego.

\subsubsection{Kolizje} w teorii rozwiązywane są prosto - jeżeli dwa obiekty zachodzą na siebie, należy je skleić. Algorytm sekwencyjny do tego celu sprawdzałby kolejne pary planet, usuwając te kolidujące ze sobą. Jeżeli jednak chcemy zrealizować obsługę kolizji równolegle, rozwiązanie naiwne okazuje się nie być prawidłowe. Gdyby każda planeta w osobnym wątku CUDA sprawdzała kolizje ze wszystkimi planetami po kolei, mogłoby dojść do konfliktów.

\paragraph{} Najpierw opiszę sposób, w jaki usuwamy planety. Przy każdej kolizji jedna planeta musi zostać usunięta, a druga staje się sumą fizyczną tych dwóch planet. Usuwanie planet przebiega dwuetapowo. Pierwszy etap, usunięcie logiczne, to zwykłe wyzerowanie masy oraz promienia usuwanej planety. Wyzerowanie masy powoduje, że planeta nie oddziałuje na inne, jest więc transparentna dla modułu fizyki. Wyzerowanie promienia natomiast "ukrywa" planetę przed modułem graficznym, który dzięki temu jej nie wyświetla. Dodatkowo, algorytm wykrywający kolizje ignoruje planety o zerowym promieniu.
Drugi etap usuwania, usunięcie fizyczne, powoduje przesunięcie wszystkich "istniejących" planet do początku tablicy - dla każdego parametru osobno. Realizowane jest to przy pomocy funkcji cudppCompact z biblioteki cudpp.
Wewnątrz tej funkcji, używana jest funkcja cudppScan, która efektywnie zamienia tablicę zerojedynkową, odróżniającą planety nieusunięte od usuniętych, na tablicę indeksów, pod które należy skopiować te planety.

\paragraph{} Wracając do problemu z równoległością w rozwiązywaniu kolizji - w implementacji naiwnej mogłoby się zdarzyć tak, że dwie planety w tym samym czasie wykryłyby kolizję z trzecią planetą, obliczyłyby parametry wynikowej planety (każda ze sobą), po czym wyzerowałyby parametry tej planety. W efekcie kolidująca planeta zostałaby "zdublowana" - w szczególności jej masa dodałaby się do obu kolidujących z nią planet.

\paragraph{} Obsługa kolizji przebiega w efekcie dwuetapowo: pierwszy etap to wykrycie kolizji, drugi to ich rozwiązanie, czyli sklejanie kolidujących planet. Poza wymienionymi wyżej czynnikami należy wspomnieć o jeszcze jednym - o klasteryzacji. Kolizje rozwiązywane są jedynie wewnątrz klastrów, gdyż zakładamy, że prawdopodobieństwo kolizji planet z różnych klastrów jest pomijalnie niewielkie.

\paragraph{Detekcja kolizji} - definiujemy tablicę kolizji k. Jeżeli planeta i nie koliduje z żadną, \ensuremath{k[i] = i}. Jeżeli natomiast wykryła kolizję z planetą j, \ensuremath{k[i] = j}. Dodatkowo wymagamy, żeby relacja ta spełniała warunek \ensuremath{i\prec j} w pewnym porządku liniowym. Dzięki temu kolidujące planety nie utworzą cyklu.

\paragraph{} Kolizje rozwiązujemy wewnątrz klastra.  Z definicji tablicy shuffle wynika, że \ensuremath{i \neq j \Rightarrow shuffle[i]\neq shuffle[j]}. Wobec tego relacja \ensuremath{shuffle[i] \prec shuffle[j] \Leftrightarrow i<j } tworzy porządek liniowy.

\paragraph{} Detekcja przebiega następująco:
\begin{lstlisting}
merge_needed = false
foreach planet p in parallel
	id = p.index + 1
	while id < count[ p.cluster ]
		if( kolizja( p, planet[ shuffle[ id ] ] ) )
			merge_needed = true
			k[ shuffle[ p.id ] ] = shuffle[ id ]
			return;
		++id
	k[ shuffle[ p.id ] ] = shuffle[ p.id ]
\end{lstlisting}

\paragraph{} W efekcie dostajemy tablicę k. Jeżeli nie została ustawiona flaga merge\_needed, kończymy. Jeżeli została, zaczynamy sklejanie planet:
\begin{lstlisting}
done = false
k_in = k
while !done
	done = true
	foreach planet p in parallel	
		if( k_in[ p.id ] == p.id )
			k_out[ p.id ] = p.id
			return
		if( k_in[ k_in[ p.id ] ] != k_in[ p.id ] )
			k_out[ p.id ] = k_in[ p.id ]
			done = false
		merge( p, planet[ k_in[ p.id ] ] )
		k_out[ p.id ] = p.id
	k_in <=> k_out
\end{lstlisting}

\paragraph{} Po czym powtarzamy detekcję.
