\documentclass{beamer}

\usepackage{polski}
\usepackage[T1]{fontenc}
\usepackage[utf8]{inputenc}

\usepackage[polish]{babel}

\usepackage{subfig}
\usepackage{rotating}

\usetheme[backgroundimagefile=kosmos.jpg,opacity=0.6]{diepen}

\title{Symulacja układu planetarnego na GPU przy użyciu CUDA i OpenGL}
\subtitle{część pierwsza}
\author{Daniel Kłobuszewski\and Jakub Kotur}
\institute{Politechnika Warszawska}
\date{\today}

\begin{document}

\frame[plain]{\titlepage}

\frame{
	\frametitle{Treść prezentacji}
	\tableofcontents
}

\section{Opis technologii}\label{sec:opis technologii}


\subsection{CUDA}\label{sub:cuda}
\frame{ \frametitle{CUDA} \tableofcontents[currentsubsection] }

\frame
{
	\frametitle{CUDA}
	\begin{itemize}
	\item{programowanie ogólnego zastosowania na kartach graficznych}
	\item{równoległe wykonanie}
	\item{maksymalne wykorzystanie sprzętu}
	\end{itemize}
}

\frame
{
	\frametitle{Macierz wątków}
	\begin{figure}
	\centering
		\includegraphics[height=7cm]{img/cuda_threads.png}
	\end{figure}
}

\frame
{
	\frametitle{Rodzaje pamięci na GPU}

	\begin{itemize}
	\item lokalna
		\item globalna
		\item współdzielona
		\item tekstur
		\item stała
		\end{itemize}
}

\frame
{
	\frametitle{Dostęp do pamięci}
	\begin{figure}
	\centering
		\includegraphics[height=7cm]{img/cuda_memory.png}
	\end{figure}
}

\frame
{
	\frametitle{Ograniczenia}
	\begin{itemize}
	\item{Prędkość transferów pamięci}
	\item{Czas dostępu do pamięci}
	\item{Ilość rdzeni na GPU}
	\item{Taktowanie GPU}
	\item{Jednoczesny dostęp do pamięci ("'coalesced"')}
	\end{itemize}
}

\subsection{OpenGL}\label{sub:opengl}
\frame{ \frametitle{OpenGL 3.2} \tableofcontents[currentsubsection] }

\frame
{
	\frametitle{Jednostki cieniowania}

	Programowalne jednostki cieniowania (shadery):

		\begin{description}
	\item[vertex] - wykonywane dla każdego wierzchołka
		\item[fragmet] - wykonywane dla każdego piksela
		\pause
		\item[geometry] - wykonywane dla każdej figury (np. trójkąt, linia)
		\end{description}

}

\frame
{
	\frametitle{Vertex Shader}
	\begin{block}{Wejście}
	\begin{itemize}
	\item pozycja (przestrzeń modelu)
		\item kolor wierzchołka
		\item koordynaty tekstury
		\end{itemize}
	\end{block}
	\begin{block}{Wyjście}
	\begin{itemize}
	\item pozycja (przestrzeń projekcji)
		\item kolor wierzchołka
		\item koordynaty tekstury
		\end{itemize}
	\end{block}
}

\frame
{
	\frametitle{Geometry Shader}

	\begin{itemize}
	\item wejście/wyjście analogiczne do vertex shadera
		\item może generować nowe wierzchołki
		\item wyjściowe figury mogą się różnić od wejściowych
		\end{itemize}
}

\frame
{
	\frametitle{Fragment (Pixel) Shader}

	\begin{block}{Wejście}
	\begin{itemize}
	\item pozycja
		\item kolor piksela
		\item koordynaty tekstury
		\end{itemize}
	\end{block}
	\begin{block}{Wyjście}
	\begin{itemize}
	\item kolor piksela
		\end{itemize}
	\end{block}
}

\frame
{
	\frametitle{Potok renderowania}
	\begin{figure}
	\centering
		\includegraphics[height=7cm]{img/potok.png}
	\label{fig:potok}
	\end{figure}

}

\subsection{Organizacja pamięci}\label{sub:organizacja pamięci}
\frame{ \frametitle{Organizacja pamięci} \tableofcontents[currentsubsection] }

\frame
{
	\frametitle{Organizacja pamięci}

	Część wspólna:
		\begin{itemize}
	\item pozycja
		\item promień
		\end{itemize}

	\pause
		Rodzaje buforów:

		\begin{itemize}
	\item bufory OpenGL
		\item bufory CUDA
		\end{itemize}

}

\begin{frame}[fragile]
\frametitle{Implementacja buforów}
\begin{verbatim}
//   * GFX
GBUF<int>    model;
GBUF<float>  emissive;

//   * COMMON
GBUF<float3>   pos;
GBUF<float>    radius;
GBUF<uint32_t> count;

//   * PHX
CBUF<float>    mass;
CBUF<float3>   velocity;
\end{verbatim}
\end{frame}

\frame
{
	\frametitle{Dane graficzne}

	Dane dla silnika graficznego trzymane w pamięci tekstury.
		\begin{itemize}
	\item tekstura modeli
		\item tekstura tekstur
		\item tekstura normalnych
		\end{itemize}
}

\subsection{Deferred rendering}\label{sub:deferred rendering}
\frame{ \frametitle{Deferred rendering} \tableofcontents[currentsubsection] }

\frame
{
	\frametitle{Deferred rendering}

	\begin{itemize}
	\item generowanie bufora geometrii
		\item bufor jest proporcjonalny do wielkości ekranu
		\item finalny obrazek jest efektem postprocesingu g-bufora
		\item efekty graficzne (np. oświetlenie) są proporcjonalne obliczeniowo do wielkości bufora
		\end{itemize}
}

\frame
{
	\frametitle{g-bufor}

	\begin{figure}
	\centering
		\subfloat[color]{\label{fig:gcol}\includegraphics[width=0.3\textwidth]{img/gbuff_col.jpg}} \hspace{.0\textwidth}
	\subfloat[głębia]{\label{fig:gdepth}\includegraphics[width=0.3\textwidth]{img/gbuff_z.jpg}} \hspace{.0\textwidth} 
	\subfloat[normalne]{\label{fig:norm}\includegraphics[width=0.3\textwidth]{img/gbuff_norm.jpg}} \hspace{.0\textwidth} \\
		\pause
		\subfloat[efekt]{\label{fig:gend}\includegraphics[width=0.3\textwidth]{img/gbuff_fin.jpg}}
	\label{fig:deferred_rednering}
	\end{figure}
	\setcounter{subfigure}{0}
}

\frame
{
	\frametitle{Wady i zalety}

	\begin{block}{Zalety}
	\begin{itemize}
	\item dobrze skalowalny ze względu na ilość dynamicznych świateł
		\item zmniejsza ilość obliczeń przy dużej ilości geometrii
		\item wydajny przy dużych otwartych przestrzeniach
		\end{itemize}
	\end{block}
	\begin{block}{Wady}
	\begin{itemize}
	\item duże bufory
		\item ogromny problem z przezroczystością
		\item wymagane jest cieniowanie per piksel
		\item na starych komputerach potrzebne jest wiele przejść do wygenerowania g-bufora (brak Multi Render Targets)
		\end{itemize}
	\end{block}
}

\frame
{
	\frametitle {Przezroczystości}

	\begin{figure}
	\centering
		\includegraphics[height=7cm]{img/necro.jpg}
	\label{fig:necro}
	\caption*{Necrovision}
	\end{figure}
}

\frame
{
	\frametitle{Przykłady użycia}

	Używany w wielu nowych grach:

		\begin{itemize}
	\item Stalker: Shadow of Chernobyl
		\item Grand Theft Auto IV
		\item StarCraft II
		\item CryEngine 3
		\end{itemize}
}

\section{Silnik graficzny}\label{sec:silnik graficzny}
\frame{ \frametitle{Silnik graficzny} \tableofcontents[currentsection] }

\subsection{Wymagania}\label{sub:wymagania}
\frame{ \frametitle{Wymagania} \tableofcontents[currentsubsection] }

\frame
{
	\frametitle{Wymagania}

	Wymagania funkcjonalne:
		\begin{description}
	\item[duże układy] - over 9000 planet
		\item[gwiazdy] - świecenie, rozmycie
		\item[komety] - warkocz pyłu
		\item[atmosfery] - półprzezroczyste powłoki wokół planet
		\end{description}
}

\frame
{
	\frametitle{Wejście}
	\begin{figure}
	\centering
		\includegraphics[height=7cm]{img/r_points.png}
	\label{fig:deferred_rednering}
	\end{figure}
}

\frame
{
	\frametitle{Wyjście}
	\begin{figure}
	\centering
		\includegraphics[height=7cm]{img/r_curr.png}
	\label{fig:deferred_rednering}
	\end{figure}

}

\subsection{Generowanie geometrii}\label{sub:generowanie geometrii}
\frame{ \frametitle{Generowanie geometrii} \tableofcontents[currentsubsection] }

\frame
{
	\frametitle{Geometria}

	\begin{figure}
	\centering
		\subfloat[60]   {\label{fig:sph0}\includegraphics[width=0.25\textwidth]{img/sph_0.png}} \hspace{.0\textwidth}
	\subfloat[960]  {\label{fig:sph2}\includegraphics[width=0.25\textwidth]{img/sph_2.png}} \hspace{.0\textwidth}
	\subfloat[61440]{\label{fig:sph5}\includegraphics[width=0.25\textwidth]{img/sph_5.png}} \hspace{.0\textwidth} \\
		\pause
		\subfloat[4]    {\label{fig:sph_def}\includegraphics[width=0.25\textwidth]{img/sph_def.png}} \hspace{.0\textwidth} 
	\label{fig:deferred_rednering}
	\end{figure}
	\setcounter{subfigure}{0}
}

\subsection{Konstrukcja silnika}\label{sub:konstrukcja silnika}
\frame{ \frametitle{Konstrukcja silnika} \tableofcontents[currentsubsection] }

\frame
{
	\frametitle{Przejścia}

	Kolejne etapy wykonywane przez silnik graficzny:

		\begin{enumerate}
	\item generowanie g-bufora
		\item światła ambient i emissive
		\item światła dynamiczne
		\item rozmycie 
		\item efekt atmosfery
		\item efekt komet
		\end{enumerate}
}

\section{Silnik fizyczny}\label{sec:silnik fizyczny}
\frame{ \frametitle{Silnik fizyczny} \tableofcontents[currentsection] }

\subsection{Problem n ciał}
\frame{ \frametitle{Konstrukcja silnika} \tableofcontents[currentsubsection] }

\frame
{
	\frametitle{Problem n ciał}
	\begin{itemize}
	\item{Nierozwiązywalny analitycznie dla n > 3}
	\item{Rozwiązania numeryczne muszą zamieniać czas ciągły na dyskretny}
	\item{A zatem - symulacja}
	\end{itemize}
}

\subsection{Klasteryzacja}
\frame
{
	\frametitle{K-means}
	\begin{itemize}
	\item{Algorytm klasteryzacji}
	\item{U nas - liczenie relacji tylko dla planet w klastrze i dla samych klastrów}
	\end{itemize}
}

\frame
{
	\frametitle{K-means - przebieg algorytmu}
	\begin{enumerate}
	\item{Znajdź k początkowych punktów oznaczających środki klastrów}
	\item{Przyporządkuj każdy punkt do najbliższego klastra (środka)}
	\item{Wyznacz nowe środki klastrów}
	\item{Jeżeli nastąpiła zmiana i błąd jest zbyt duży, przejdź do punktu 2.}
	\end{enumerate}
}

\frame
{
	\frametitle{K-means - przebieg algorytmu}
	\begin{figure}
	\centering
		\subfloat{\includegraphics[width=0.35\textwidth]{img/kmeans1.png}} \hspace{0cm}
	\subfloat{\includegraphics[width=0.35\textwidth]{img/kmeans2.png}} \hspace{0cm}
	\pause
		\subfloat{\includegraphics[width=0.35\textwidth]{img/kmeans3.png}} \hspace{0cm}
	\subfloat{\includegraphics[width=0.35\textwidth]{img/kmeans4.png}} \hspace{0cm}
	\end{figure}
}

\subsection{Kolizje}
\frame
{
	\frametitle{Kolizje}
	\begin{itemize}
	\item{Zderzenia sprężyste}
	\item{Zderzenia niesprężyste}
	\end{itemize}
}

\frame
{
	\frametitle{Koniec}
	\begin{figure}
	\centering
		Dziękujemy za uwagę
		\end{figure}
}

%\setbeamercolor{normal text}{bg=black}

%\setbeamertemplate{background}{\includegraphics[width=\paperwidth]{img/mirabella.jpg}}
%\frame { }

%\setbeamertemplate{background}{\includegraphics[width=\paperwidth]{img/sokol_maltanski.jpg}}
%\frame { }

\end{document}

