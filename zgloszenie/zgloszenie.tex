\documentclass[a4paper,titlepage,10pt]{article}

\usepackage[T1]{fontenc}
\usepackage{cmbright}
%\usepackage[OT4]{fontenc}
\usepackage[utf8]{inputenc}
\usepackage{polski}

\usepackage{enumerate}
\usepackage{amssymb}
\usepackage{amsmath}
\usepackage[pdftex]{graphicx}
\usepackage{tikz}
\usepackage[colorlinks=true,linkcolor=blue]{hyperref}
\usepackage{listings}

\usepackage[a4paper, top=2.5cm, bottom=2.5cm]{geometry}
\linespread{1.3}
\lstset{frame = single, numbers = left, basicstyle = \footnotesize, stepnumber = 5}

\title{\huge Mutant Cyborg Wars \\\large Dokumentacja Funkcjonalna\\\small Metody Sztucznej Inteligencji }
\author{Daniel Kłobuszewski\and Jakub Kotur}

\begin{document}

	\paragraph{}

	Wydział Matematyki i Nauk Informacyjnych

	\paragraph{}
	\begin{center}
	ZGŁOSZENIE TEMATU PRACY DYPLOMOWEJ INŻYNIERSKIEJ

	Na rok akademicki 2010/2011
	\end{center}

	\begin{tabular}{ | l | p{.5\textwidth} | }
	\hline
	Imię, nazwisko, tytuł, stopień naukowy & dr inż. Krzysztof Kaczmarski \\ \hline
	Zakład, telefon, e-mail & ZZIiMN, k.kaczmarski@mini.pw.edu.pl \\\hline
	Tytuł zgłoszanej pracy & Symulacja układu planetarnego na GPU przy użyciu CUDA i OpenGL \\\hline
	Kierunek & INFORMATYKA \\\hline
	Imię i nazwisko dyplomanta/ki & Daniel Kłobuszewski, Jakub Kotur \\\hline
	\end{tabular}

	\paragraph{Tematyka zgłoszonej pracy}

	\paragraph{}
	
	Praca ma na celu wykonanie symulacji układów planetarnych z uwzględnieniem zależności grawitacyjnych, oraz zaawansowaną wizualizację. Aby osiągnąć możliwie duże układy planetarne obliczenia będą wykonane na procesorach wielordzeniowych GPU z wykorzystaniem technologii CUDA (compute unified device architecture). Wizualizacja ma natomiast uwzględniać takie zjawiska jak gwiazdy (obiekty świecące), komety (warkocze ognia) oraz atmosfery otaczające planety. Aby zbędnie nie spowalniać symulacji, całość obliczeń w miarę możliwości powinna odbywać się na karcie graficznej, bez kopiowania danych do jednostki centralnej.

%        \paragraph{}
%        
%        W wyniku ma powstać pogram który będzie możliwie realistycznie przedstawiał układy planetarne, ze szczególnym uwzględnieniem układu słonecznego, oraz okolicznych układów. 

	\paragraph{Podział pracy}
	
	
\end{document}

